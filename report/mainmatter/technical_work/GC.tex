\section{Garbage Collector}
\label{sec:gc}
Automatic garbage collection is the process of looking at heap memory, identifying which objects are in use and which are not, and deleting the unused objects. An in use object, or a referenced object, means that some part of your program still maintains a pointer to that object. An unused object, or unreferenced object, is no longer referenced by any part of your program. So the memory used by an unreferenced object can be reclaimed.

In a programming language like C, allocating and deallocating memory is a manual process. In Java, process of deallocating memory is handled automatically by the garbage collector.\cite{oracleGC}\\
There are 4 different types of garbage collectors available for Java
\subsection{Serial Garbage Collector}
The serial collector is the default for client style machines in Java SE 5 and 6. With the serial collector, both minor and major garbage collections are done serially (using a single virtual CPU). In addition, it uses a mark-compact collection method. This method moves older memory to the beginning of the heap so that new memory allocations are made into a single continuous chunk of memory at the end of the heap. This compacting of memory makes it faster to allocate new chunks of memory to the heap.\cite{oracleGC}
\paragraph{Usage Cases}
The Serial GC is the garbage collector of choice for most applications that do not have low pause time requirements and run on client-style machines. It takes advantage of only a single virtual processor for garbage collection work (therefore, its name). Still, on today's hardware, the Serial GC can efficiently manage a lot of non-trivial applications with a few hundred MBs of Java heap, with relatively short worst-case pauses (around a couple of seconds for full garbage collections).

Another popular use for the Serial GC is in environments where a high number of JVMs are run on the same machine (in some cases, more JVMs than available processors!). In such environments when a JVM does a garbage collection it is better to use only one processor to minimize the interference on the remaining JVMs, even if the garbage collection might last longer. And the Serial GC fits this trade-off nicely.

Finally, with the proliferation of embedded hardware with minimal memory and few cores, the Serial GC could make a comeback.\cite{oracleGC}
\paragraph{Command Line Switch}
To enable the Serial Collector we can use:
\textit{-XX:+UseSerialGC}
\subsection{Parallel Garbage Collector}

The parallel garbage collector uses multiple threads to perform the young genertion garbage collection. By default on a host with N CPUs, the parallel garbage collector uses N garbage collector threads in the collection. The number of garbage collector threads can be controlled with command-line options:
\textit{-XX:ParallelGCThreads=<desired number>}

On a host with a single CPU the default garbage collector is used even if the parallel garbage collector has been requested. On a host with two CPUs the parallel garbage collector generally performs as well as the default garbage collector and a reduction in the young generationgarbage collector pause times can be expected on hosts with more than two CPUs. The Parallel GC comes in two flavors.\cite{oracleGC}
\paragraph{Usage Cases}
The Parallel collector is also called a throughput collector. Since it can use multilple CPUs to speed up application throughput. This collector should be used when a lot of work need to be done and long pauses are acceptable. For example, batch processing like printing reports or bills or performing a large number of database queries.\cite{oracleGC}
\subparagraph{The Parallel collector}
is also called a throughput collector. Since it can use multilple CPUs to speed up application throughput. This collector should be used when a lot of work need to be done and long pauses are acceptable. For example, batch processing like printing reports or bills or performing a large number of database queries.\cite{oracleGC}
To enable the Parallel collector we can use:
\textit{-XX:+UseParallelGC}
\subparagraph{Parallel Old Collector}
To enable this garbage collector we use \textit{-XX:+UseParallelOldGC}. With this option, the GC is both a multithreaded young generation collector and multithreaded old generation collector. It is also a multithreaded compacting collector. HotSpot does compaction only in the old generation. Young generation in HotSpot is considered a copy collector; therefore, there is no need for compaction.

Compacting describes the act of moving objects in a way that there are no holes between objects. After a garbage collection sweep, there may be holes left between live objects. Compacting moves objects so that there are no remaining holes. It is possible that a garbage collector be a non-compacting collector. Therefore, the difference between a parallel collector and a parallel compacting collector could be the latter compacts the space after a garbage collection sweep. The former would not.\cite{oracleGC}
\subsection{Concurrent Mark Sweep (CMS) Collector}
The Concurrent Mark Sweep (CMS) collector (also referred to as the concurrent low pause collector) collects the tenured generation. It attempts to minimize the pauses due to garbage collection by doing most of the garbage collection work concurrently with the application threads. Normally the concurrent low pause collector does not copy or compact the live objects. A garbage collection is done without moving the live objects. If fragmentation becomes a problem, allocate a larger heap.

Note: CMS collector on young generation uses the same algorithm as that of the parallel collector.\cite{oracleGC}
\paragraph{Usage Cases}
The CMS collector should be used for applications that require low pause times and can share resources with the garbage collector. Examples include desktop UI application that respond to events, a webserver responding to a request or a database responding to queries.\cite{oracleGC}
\paragraph{Command Line Switch}
To enable the Concurrent Mark Sweep (CMS) collector we can use:
\textit{-XX:+UseConcMarkSweepGC}\\
And to set the number of threads use: \textit{-XX:ParallelCMSThreads=<n>}
\subsection{G1 Garbage Collector}
The Garbage First or G1 garbage collector is available in Java 7 and is designed to be the long term replacement for the CMS collector. The G1 collector is a parallel, concurrent, and incrementally compacting low-pause garbage collector that has quite a different layout from the other garbage collectors described previously.\cite{oracleGC}
\paragraph{Command Line Switch}
To enable the Concurrent Mark Sweep (CMS) collector we can use:
\textit{-XX:+UseG1GC}